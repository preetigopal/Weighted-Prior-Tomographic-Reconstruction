\documentclass{article}
\bibliographystyle{IEEEtran}

\usepackage{amsmath}
\usepackage{graphicx}
\usepackage{caption}
\usepackage{subcaption, tabularx}
\usepackage{xcolor}
\usepackage{boxedminipage}
\usepackage{a4wide}
\usepackage{subcaption,comment}
\usepackage[colorlinks=true]{hyperref}
\title{Response to Reviewers and Associate Editor}
\begin{document}
\date{}
\maketitle

This paper was submitted earlier, and out of the three reviewers, two
of them had pronounced this as ``publish unaltered'' and one reviewer
(R1) did not recommend resubmission. The handling Editor has
encouraged a resubmission of this paper taking into account the
comments of R1.  We thank all reviewers for their remarks and in this
document, we explain the steps we have taken to address the reviews
received in our previous submission to TCI. R1 makes the following
comments, and we quote excerpts in blue here:
\begin{quote}
  \textcolor{blue}{The authors have spent considerable effort in revising the
  manuscript and addressing the concerns of the reviewers. I would
  like to thank the authors for their effort. I believe the manuscript
  has improved because of these revisions, especially in the structure
  of the paper and the discussion of related work. Unfortunately, some
  of my main concerns with regards to the experiments, the accuracy of
  the proposed method, and the presentation of the findings remain
  present in the revised manuscript. In some cases the revisions have
  even increased my concerns about certain aspects. Mainly, the
  included experiments still do not clearly show that the proposed
  method improves upon existing ‘simple’ prior-based methods
%% , and the
%%   new results even show that there is evidence that FDK is more
%%   accurate than the proposed method exactly in the region that changed
  }
  \end{quote}

The simpler method is based on the well-known ``Total Variation''. 
In particular, we address the following main concern of the reviewer 
\begin{quote}  
    \textcolor{blue}{Note that the total
variation minimization results are much better than the l1-ls results that the authors show in the
paper}
\end{quote}

    by rewriting the paper considerably.  We believe the paper
    addresses this (correct) stinging criticism of R1.  In fact R1 has taken
    pains in showing how the Total-Variation method (and sometimes
    FDK) is better than our earlier Compressed Sensing plus
    spatially-varying prior-based reconstruction, and we gratefully
    acknowledge the same.

Again, this is correct for the dataset shown, and we accept this to be
true.

  We re-wrote our code and upon further analysis observed that TV is
  indeed better suited as a baseline reconstruction method for our
  datasets and our subsampling choices. Hence, our current paper
  presents all results using TV regularization coupled with
  spatially-varying technique, and demonstrates  benefits over
  TV-only and backprojection-only methods.  \emph{We have eliminated our
  previous implicit claim that the CS based scheme is optimal for the
  dataset provided. }

  As far as FDK is concerned, the current results indeed show the
  superiority of the TV-based scheme (as also noted by the reviewer).

    That said, the goal of our previous and curent work is to provide
    a technique that can improve upon a chosen baseline reconstruction
    when measurements are extremely sparse.  One may choose any
    baseline reconstruction (including an inferior -- for this dataset
    -- Compressed sensing prior) for pedagogical, historical, or
    commercial reasons, and given that, our key question is -- is
    there a way to improve?  The spatially-varying method of
    moderating the prior settles this question.

We also take this opportunity to address other comments given by R1
and the handling editor.

 \begin{enumerate}
 \item    \textcolor{blue}{...the paper should at least include a (short) discussion about the
     possibility of using deep learning for this, and its disadvantages.}

   We accept this point, and have now discussed the difficulties in
   using deep-learning techniques for the current problem in Section.~7C
   of our paper. 

\item \textcolor{blue}{There are no added comparisons with popular existing methods for
  reconstruction with a small number of projections (TV, algebraic techniques, etcetera)}\\
  
 Among the family of algebraic techniques, we have compared our method
 with Algebraic Reconstruction Technique (ART)~\cite{art} and Simultaneous
 Algebraic Reconstruction Technique (SART)~\cite{sart}. In order to use the Discrete Algebraic Reconstruction Technique (DART)~\cite{dart}, one must assume the imaged object to be made of only a few fixed number of known attenuation coefficients. Since we do not make any such assumption about  our datasets, we have not used DART. %% IS THERE A NEED TO HIGHGLIHT what we have not done?  Also give references?
 Among other iterative techniques, we
 have used Simultaneous Iterative Reconstruction Technique (SIRT)~\cite{sirt},
 Total Variation (TV)~\cite{TV},~\cite{TVReg} and Compressed Sensing (CS)~\cite{lasso} with Haar wavelet
 and DCT as sparsity basis. The details of solvers used for these
 algorithms and their reconstructions are in Page.~7 (Section-6A) of
 the paper. For CS reconstruction, we observed that Haar wavelet and
 DCT were best suited for our dataset among various other possible
 sparsity basis.

\item \textcolor{blue}{Why did the authors not choose the standard
  settings for SSIM?...I don’t understand the reason given for not
  including RMSE: the values don’t have to be optimized again -- you
  could just optimize for SSIM but at least still report RMSE
  comparisons.}

We accept the criticism, and have now presented both RMSE and SSIM
values for all reconstructions in Tables 1-5 of the main paper. 
  \begin{comment}
Since
RMSE corresponds to error in intensity values, we have used SSIM with
a higher weightage to structure for the purpose of measuring error in
preservation of structures in our reconstructions.

  The reason for not reporting RMSE stems from the optimization
  process in Equation~\ref{Eq:simple_TV}.
  
 \begin{equation}
   J_{TV}(\boldsymbol{x}) = \lVert\boldsymbol{\mathcal{R}x}- \boldsymbol{y}\rVert_2^2 + \lambda_{TV}TV(\boldsymbol{x})
   \label{Eq:simple_TV}
 \end{equation}

  This optimization routine 
  transforms the reconstruction values to a different range of
  numerical values.  That is, in our experiments, we observed that for
  a given dataset and a set of projection measurements, the intensity
  span of histograms of reconstructed volumes differ considerably across various
  methods and solvers.  As a result, the RMSE values are not
  particularly useful, and may even be misleading.  We provide the RMSE numbers below to show the
  improvement trend but we believe it does not serve the interest of
  the community to report these in the main paper.

  For the same reason, the standard setting of SSIM in matlab (for
  instance) emphasizes the intensity. We found that even when the
  quality of reconstruction is acceptable visually, the SSIM values
  with the standard values were not acceptable.  We choose
  to focus on preserving the structures on our reconstructions, and
  not rely on the absolute intensity values alone. Hence, we choose
  SSIM with a higher weightage to structure-preservation.

  For the record, sample RMSE values are presented
  here in Table~\ref{table:rmse} that shows
  the superiority of our method at the risk of misleading absolute numbers.

  %% Why not combine all of the values in one table?
\end{comment}

\item \textcolor{blue}{A discussion about computational costs (time,
  memory) should be included in the paper.}
  
We accept this point, and present details in Section.~7B of our paper.
We also provide the code to reproduce the compute time claimed.

\end{enumerate}

\bibliography{tci_ref}  
\end{document}

